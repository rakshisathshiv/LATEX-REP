\documentclass[a4paper, 12pt]{report}
\usepackage{graphicx} 
\usepackage{ragged2e}
\usepackage{xcolor}
\usepackage{fancyhdr} 
\fancyhead{} 
\fancyhead[L]{left header} 
\fancyhead[R]{right header \quad \thepage} 
\fancyfoot{} 
\fancyhead[l]{LATEX}
\fancyhead[r]{Rakshitha G S}
\fancyhead[C]{PROGRAM 12}
\fancyfoot{} 
\fancyfoot[L]{Dept of ISE,CIT GUBBI} 
\fancyfoot[R]{2023-2024 \quad \thepage} 
\pagestyle{fancy} 
\begin{document}
\include{FrontPage-VTU}
\include{Certificate-VTU} 
\tableofcontents
%\pagenumbering{arabic} 
%\setcounter{page}{1} 
\thispagestyle{empty}
\chapter{Introduction}
\pagenumbering{arabic} 
\setcounter{page}{1} 
\section{Context based Diversification}
The human era is evolved and dominated through the ultimate intention to know about the Universe and its assets more and more. This in turn persuaded him to gather the immense information of need in the form of theory, tools, intuitions, visuals, and ultimately as the form of abstract objects \cite{ffy}, \cite{xyshw}, \cite{tbhs}, \cite{dpgk}, \cite{jaip}, \cite{pdkgvym}. 

Deciding the context of the search query based on its representation over a concept network using fuzzy methods provides a better thrust to the overall search process. The existing context based search diversification process emphasizes the importance of the numerical representation of the query over a data repository \cite{fpsu}, \cite{sysxh}, \cite{zwjfw}, \cite{dwhh}. The search operation can use these semantically meaningful segments as a confident segment in the conceptual network.  
\chapter{Literature Review}
Corresponding Literature works 
\chapter{Methodology} 
Corresponding Methodology
\chapter{Performance analysis} 
Corresponding Performance analysis 
\renewcommand{\thefigure}{4.1}
\begin{figure}[htbp] 
\centerline{\includegraphics[scale=.8]{CN}}
\caption{A portion of concept network for query "hotel California".} 
\label{fig}
\end{figure} 
\chapter{Conclusion} 
Corresponding Conclusion 
\bibliographystyle{IEEEtran} 
\bibliography{prg9} 
\end{document} 